\documentclass[a4paper,14pt]{article}

\title{report}
\author{Зарубин Всеволод}
\date{today}
\usepackage[T2A]{fontenc}
\usepackage[utf8]{inputenc}
\usepackage[english,russian]{babel}
\usepackage{amsmath,amsfonts,amssymb,amsthm,mathtools,bpchem}
\usepackage{fancyhdr}
\usepackage{indentfirst}
\usepackage{float}
\usepackage{etaremune}

\usepackage[margin=1in]{geometry}

\pagestyle{fancy}
\fancyhf{}
\rhead{\thepage}
\renewcommand{\headrulewidth}{0pt}

\thispagestyle{empty}


\begin{document}
\begin{titlepage}
\begin{center} 
 
\large Московский физико-технический институт\\
Факультет молекулярной и химической физики\\
\vspace{6.5cm}
\Large Задание по курсу \\ Вычислительная математика\\
\textbf{\Large Гиперболические системы уравнений}\\
\end{center} 

\vspace{7cm}
{\par 
	\raggedleft \large 
	\emph{Выполнил}\\ 
	студент 3 курса\\ 
	643 группы ФМХФ\\ 
	Зарубин Всеволод 
\par}
\begin{center}
\vfill \today
\end{center}
\end{titlepage}
\newpage
\setcounter{page}{2}

\newpage
	\section{Постановка задачи}
    \Large Необходимо численно решить систему линейных \\гиперболических уравнений.\\
    \subsection{Параметры задачи}
    Используются разностные схемы:
    \begin{center}
    \begin{enumerate}
        \item Схема Куранта-Изаксона-Риса
        \item Гибридная схема Федоренко. \\
        Порядок схемы изменяется от второго в областях\\
        гладкости решения, до первого в областях больших градиентов.  
    \end{enumerate}
    \end{center}
    \vspace{0.5cm}
    Модельное уравнение:\\
    \begin{equation}
        \frac{\partial \mathbf{u}}{\partial t}+\mathbf{A} \frac{\partial \mathbf{u}}{\partial x} = 0,     a = const > 0
    \end{equation}
    \vspace{0.5cm}
    Сетка:
    \begin{center}
    $xm = mh, m = 0..M, Mh = 1; t^n = n\tau, n = 0..N, N\tau = 1$\\
    \end{center}
    \vspace{1cm}
    \subsection{Рассматривается следующая система линейных\\ гиперболических уравнений} 
    \begin{equation}
    \frac{\partial \mathbf{u}}{\partial t}+\mathbf{A} \frac{\partial \mathbf{u}}{\partial x}=\mathbf{b}(x), 0 \leqslant x \leqslant 1,0 \leqslant t \leqslant 1, \quad \mathbf{u}(x, 0)=\left( \begin{array}{c}{x^{3}} \\ {1-x^{2}} \\ {x^{2}+1}\end{array}\right)
\end{equation}
    
    \vspace{0.5cm}
    \begin{equation}
\mathbf{A}=\left( \begin{array}{ccc}{-4} & {\frac{3}{5}} & {\frac{128}{5}} \\ {-1} & {\frac{-36}{5}} & {\frac{-16}{5}} \\ {1} & {\frac{1}{5}} & {\frac{-19}{5}}\end{array}\right), \mathbf{b}(x)=\left( \begin{array}{l}{0} \\ {0} \\ {0}\end{array}\right)
\end{equation}
    
    \newpage
    \subsection{Конкретизация задачи}
    \begin{itemize}
    \item Привести систему к характеристическому виду,\\ предложить корректную постановку граничных условий.
    \item Решить численно систему уравнений с использованием указанных разностных схем.
    \item Определить характер преобладающей ошибки.
    \item Определить, монотонна ли схема.
    \item Оценить апосториорно порядок сходимости схем.
    \end{itemize}
    \vspace{1cm}
    \section{Решение задачи}
    \subsection{Приведение системы к характеристическому виду}
    Выполним спектральное разложение матрицы $A$ системы:
    \begin{equation}
    A=P D P^{-1}
    \end{equation}
    \vspace{0.5cm}
    \Large Справка:\\
        Спектральное разложение матрицы — это представление квадратной матрицы A в виде произведения трёх матриц, {\displaystyle A=V\Lambda V^{-1}} A=V\Lambda V^{{-1}}, где {\displaystyle V} V — матрица, столбцы которой являются собственными векторами матрицы  A, {\displaystyle \Lambda } \Lambda  — диагональная матрица с соответствующими собственными значениями на главной диагонали, {\displaystyle V^{-1}} V^{{-1}} — матрица, обратная матрице {\displaystyle V} V.
    \vspace{0.5cm}
    С учётом разложения система принимает вид:
    \begin{equation}
    \frac{\partial \mathbf{u}}{\partial t}+\mathbf{P D P^{-1}} \frac{\partial \mathbf{u}}{\partial x}=0
    \end{equation}
    \vspace{0.5cm}
    Домножая слева на $P^{-1}$, получим систему в характеристическом виде:
    \begin{equation}
    \frac{\partial \mathbf{R}}{\partial t}+\mathbf{D} \frac{\partial \mathbf{R}}{\partial x}=0
    \end{equation}
    где $\mathbf{R}=\mathbf{P^{-1}u}$
    \vspace{0.5cm}
    В итоге получаем следующую систему уравнений:
    \begin{equation}
    \begin{cases} \frac{\partial R_1}{\partial t} -9\frac{\partial R_1}{\partial x}=0 \\ \frac{\partial R_2}{\partial t} -7\frac{\partial R_2}{\partial x}=0 \\ \frac{\partial R_3}{\partial t} + 1\frac{\partial R_3}{\partial x}=0 \end{cases}
    \end{equation}
    \newpage
    Начальные условия для инвариантов Римана:
    \begin{equation}
    \mathbf{R}(x, 0) = \left( \begin{array}{c}{\frac{-5x^{3}+25x^{2}+23}{50}} \\ {2} \\ {\frac{5x^{3}+25x^{2}+27}{50}}\end{array}\right)
    \end{equation}
    \vspace{0.5cm}
    Восстановление естественных переменных возможно по формуле:
    \begin{equation}
    \mathbf{u}=\mathbf{PR}
    \end{equation}
    
\end{document}